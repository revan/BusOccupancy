\documentclass[12pt,journal,compsoc]{IEEEtran} % Set to 11pt if we ever have the luxury of doing so.
\usepackage[T1]{fontenc}
\usepackage{graphicx}
\usepackage[cmex10]{amsmath}
\usepackage{stfloats}
\usepackage[section]{placeins}
\usepackage{minted}
\usemintedstyle{tango}
\graphicspath{ {img/} }

% IMPORTANT:
%  This document class uses logical paragraphs rather than manual ones.
%  In other words, use a blank line instead of \\ to start a new paragraph.

\newcommand{\minifig}[3]{
  \begin{figure}[!t]
    \includegraphics[width=\columnwidth]{#1}
    \caption{#2}
    \label{#3}
  \end{figure}
}

\newcommand{\bigfig}[3]{
  \begin{figure*}[!t]
    \includegraphics[width=\textwidth]{#1}
    \caption{#2}
    \label{#3}
  \end{figure*}
}

\newcommand{\biglst}[4]{
  \begin{listing*}[!t]
    \inputminted{#1}{#2}
    \caption{#3}
    \label{#4}
  \end{listing*}
}

\begin{document}
\title{Programatically Predicting [Bus] Occupancy\\
{\large Undergraduate Independent Research Report}}
\author{Daniel~Bordak, Erin~Corrado, Revan~Sopher, and Ashley~Weaver}

\IEEEtitleabstractindextext{% Comment to prevent unwanted space
\begin{abstract}
In this study we address the problem of programatically predicting bus occupancy.
We approach this in two ways: by using publicly available GPS positions to calculate stop time, and by using promiscuous Wi-Fi to count devices.
While the stop time approach proved unfruitful due to the significant time requirement in establishing a benchmark by which to interpret the data, the Wi-Fi approach yielded an accurate model once we compensated for noise.
We conclude by suggesting modifications to the experiment, and the application of this methodology to the determination of occupancy of other locales.
\end{abstract}}

\maketitle

%TODO: Reference *every* figure at least once in the text.

\section{Motivation}

\IEEEPARstart{D}{ue to} the periodic nature of class schedules, students at Rutgers tend to move between campuses during certain known time intervals during the day.
This leads to predictable migratory patterns.
However, the bus schedule does a poor job of taking student schedules into account -- for example, on Livingston campus, buses on the route to College Avenue can often be found lined up at the stops consecutively, while the buses on the route to Busch are much rarer.
To make matters worse, the Busch buses are usually packed to capacity between class periods.
The aim of this project is to develop a system for better predicting the demand for buses, such that an empirical argument for the optimization of the bus schedule might be made.

\section{Approach}

In order to facilitate simultaneous contributions from all four team members, we approached the problem in two manners: by directly collecting data about number of riders vs. stop time, and by measuring number of devices on the bus.
Two students were assigned to work on each approach.
Ultimately, we planned to combine results into a single prediction method.

\subsection*{Stop Length Inference}

In this approach, we planned to establish a correlation between the length of time a bus spends at a stop and the number of passengers currently aboard.
Our initial theory was that a bus nearing capacity would take longer at stops due to the larger number of riders entering and exiting.
Every Rutgers bus periodically reports its position, velocity, and destination to an online service -- nextbus.com's Public API.
This service is free to use, its information is publicly available. %TODO: Add example XML
As such, using it would not require any hardware or observers on the bus, allowing for cheap, large-scale execution.
We first tried to measure the strength of this correlation; if we could prove a strong positive correlation, we could use the bus travel data to measure length of stops, and therefore predict occupancy.
By running the analysis on a log of travel data, we could predict occupancy across a significantly larger domain and for theoretically no cost.

In order to establish a correlation between stop length and bus occupancy, we followed a bus on its circuit, recording data about its stops. Specifically, we recorded the number of people boarding or departing the bus, the length of the stop, and what time it was.

\subsection*{Promiscuous WiFi}

\logfig{text}{bus.dat}{Example tcpdump output\protect\footnotemark}{fig:tcpdump}

In this approach, we used the inherent correlation between the number of smartphones on the bus and the number of passengers; it is reasonable to assume that more devices means more passengers, since the devices travel with the passengers.
We hold that this correlation is fundamental enough to not require investigation, unlike the other approach.

Data collection was performed with a laptop running Linux, equipped with a typical 802.11n wireless card and a GPS sensor.
The computer ran the wireless card in monitor mode.
In normal operation, a wireless card will receive every packet in the vicinity, but discard those that are not specifically addressed to it; in monitor mode all received packets are transmitted to the operating system regardless of intended recipient. With this setup, we ran the following command:
\mint{bash}|tcpdump -e|
This command logs the headers of every packet that the wireless card sees, thus uniquely identifying every active device in the area.
The strength of signals is also recorded, so we can choose to ignore packets below a certain decibel strength.
We rode a bus along its route, logging packets and stop times.

\footnotetext{
  This logfile has been modified from its original version.
  It has been formatted to fit this page.
  The original output had one line per packet.
  Additionally, MAC addresses have been censored to protect the privacy of those we observed.
}

\section{Graph Production}

\subsection*{Preprocessing}

\logfig{json}{fancybus.json}{Example json output, abbreviated}{fig:json}

The raw output from running tcpdump contains many irrelevant details -- fields like the type of packet, or the antenna used.
As such, our first step in graph production was to preprocess this data, preserving only the information relevant to our goal.
The main data of interest was the list of MAC addresses seen in a given packet.
Each of these addresses can take one of four roles: Source, Destination, Transmitter, and Receiver.
Every packet will have some, but not all, of these types of addresses.
Since each line of tcpdump raw output represents a packet, our log files are first parsed line by line, each line individually stored as a Python Dict -- otherwise known as a Hash Map.
These Dicts hold the four potential addresses, the packet's strength, and what time it was sent at (in microseconds).
All the other information was disregarded; ignoring information during preprocessing is the simplest way of removing it from the data.

A few special considerations needed to be made during this step:
\begin{itemize}
\item Packets may not have a strength field.
  To make data handling easier later on, these are assigned a strength of zero.
  This is easier since it can be filtered out later simply by checking that the strength field is nonzero; a zero strength appearing naturally is impossible. 
\item Some lines contain only the address of a router, with no particular destination in mind.
  These are broadcast packets, sent to notify devices of the router's existence.
  Since we are monitoring personal devices, not routers, these lines are ignored, and the address identified as a router is flagged for future filtering.
\item Since MAC addresses are usually unique device identifiers, they could potentially be used to identify device owners.
  This kind of risk could subject our study to human subject regulations.
  In order to avoid this risk, addresses are saved into the Dicts as index numbers (starting from zero) rather than the actual address.
  This way, we can still differentiate between multiple devices without having to worry about our data being used for morally questionable purposes.
\end{itemize}

At the end of the parsing process, the Dicts are dumped into a JSON output file.
JSON is a open standard data interchange format.
Using JSON made utilizing the data much easier when it came to producing the graphs; most languages, Python included, have a standard library implementation for JSON, eliminating the effort spent in re-parsing the data. % TODO: Add example JSON output -- re-run raw2json with prettyprint on.

\subsection*{Filtering}
The preprocessing done above is only performed once per dataset.
Therefore, any sort of filtering that might change between two graphs must be done immediately before the construction of the graph.
We built four filters to be run at this stage:
\begin{itemize}
\item The Strength filter removes packets below a given strength threshold.
  For example, a strength filter of -50 would filter out packets of -51dB.
\item The Zero-Strength filter removes packets which have a strength field of zero.
  These packets correspond to tcpdump output lines missing a strength field, as mentioned above in the preprocessing section.
\item The Router filter removes any addresses which were flagged as routers during the parsing step.
  It also removes any packets which would no longer contain any addresses after the previous is performed.
\item The End Time filter removes any packets past a certain time.
  This is desirable when you only want to look at a part of a dataset.
  We never produced a corresponding Start Time filter since the need never arose.
\end{itemize}

\subsection*{Graphing}
For the construction of the graphs, a few additional libraries were needed: Numpy, Pandas, and Matplotlib.
Numpy is an advanced mathematical library for Python, and Pandas is a library built on top of Numpy to provide high-performance data structures.
We used these for data analysis.
Matplotlib is a 2D plotting library, which we used to generate the graphs. 
In order to simplify the graph production process, we made individual scripts for each graph type, plus a meta-script to make a united interface.
This meta-script reads a JSON file obtained from the preprocessing step, imports the data into a Pandas DataFrame (which can be thought of as a complicated 2D array), and sends this off to be filtered, as explained in the filtering step.
After the data has been filtered, it is handed off to one of the six plotting scripts:
\begin{itemize}
\item The Packet and Unique plots were created by generating an array of 3-second bins and adding the appropriate statistic to the bins.
  In the case of the packet plot, this was simply the number of packets found over those three seconds; the bin is simply an integer.
  In the case of the unique plot, however, the bin is a mathematical set holding the MAC addresses seen over the three second period.
  After the entirety of the three seconds are parsed, the bin is assigned the length of the set.
\item The First-Last Grid, Packet Histogram, and Vector plot were all created by checking the data set for each address.
  The first time each MAC address is encountered, it is added to a list with an initial value of the time it was seen as both its first and last time seen, and seen-count of one.
  Each additional time an address is seen, its last time seen value is updated to the current time, and its seen-count is incremented.
  In the Grid, the addresses are graphed using the first time seen as the x-value and the last time seen as the y-value.
  In the packet histogram, packets which fulfill the requirement of \(t_{first} + t_{max} - t_{last} < 120\) are selected, where \(t_{max}\) is the total observation time.
  These selected packets are then graphed with their y-values set as their seen-count.
  To make the graph easier to interpret, the y-axis was made to be logarithmic, and the addresses are sorted by their y-values.
  For the Vector plot, first an array of zeros was initialized with length equal to the length of observation in seconds.
  For each packet, every value between \(t_{first}\) and \(t_{last}\) was incremented.
  The resulting array was plotted as a step function, with the x-axis being its indices.
\item The segment plot is the most complicated plot we generated.
  First, a list of mathematical sets are generated, one for each stop.
  These sets are called the stop-sets.
  The stop-sets are populated with the MAC addresses seen during the corresponding stop.
  Next, a second list of mathematical sets are generated, again, one for each stop.
  These sets are called the intersect-sets.
  The intersect-sets are filled with the intersect between the stop-set of the same index value, and every other stop-set, combined by union.
  For instance, intersect-set 2 would contain
  \begin{multline*}
    \left(stop_2 \cap stop_0\right) \cup \left(stop_2 \cap stop_1\right) \cup \\
    \left(stop_2 \cap stop_3\right) \cup \dots \cup \left(stop_2 \cap stop_n\right)
  \end{multline*}
  The length of each intersect-set is considered to be the number of people on the bus at the stop.
  A special exception is made for the intersect of stop-set 0 and stop-set n.
  This set represents addresses present both at the beginning of the ride and the end.
  No normal student would do this -- this would consist only of ourselves and possibly devices present around this stop.
  As such, this set is instead subtracted from all other intersect sets; it is likely garbage.
  Afterward, we add a flat value of 2 to every stop, representing the two of us we filtered out.
\end{itemize}
Some graphs also had the actual number of people on the bus, as observed, overlaid.
This overlay was generated with a simple step function, using the value at n+1 as the value for the line between n and n+1.

\section{Results}

\minipic{occupancy}{Bus Occupancy as a function of Stop Length: stop length has no correlation to occupancy.}{fig:occ}
%TODO: Clarify whether this is for the following or preceding stop. I don't actually remember.
\minipic{netdelta}{Net Delta Occupancy as a function of Stop Length: longer stop lengths signify either positive or negative changes.}{fig:netdelta}
\minipic{absdelta}{Absolute Delta Occupancy as a function of Stop Length: strong correlation between magnitude of change and stop length. We can predict changes if we can show certain bus stops tend to positive or negative change.}{fig:absdelta}
\minipic{onestopcrop}{Net Delta Occupancy as a function of Stop Length at a Single Stop: this supports the idea that we can characterize bus stops by either positive or negative change.}{fig:onestop}
\subsection{Stop Length Inference}

Based on the experimental procedure described in TODO, Figure~\ref{fig:occ} depicts a lack of significant correlation between stop length and occupancy.
This makes sense, because the number of sedentary or otherwise intransient riders does not cause slowdowns; rather, the relevant metric is the net occupancy change; the number of people entering or exiting the bus at any given stop.
Plotting the net occupancy change as a function of stop length (Fig.~\ref{fig:netdelta}), however, was also inconclusive, as the longer stops could signify either positive or negative changes.
From stop timings alone, we would not be able to predict the direction of change.

Plotting absolute change (fig.~\ref{fig:absdelta}), however, shows a much stronger correlation than either of the previous graphs.
Absolute change is not a useful metric on its own, but if it could be proven that certain bus stops tended to have a generally positive or negative net change, we could predict the sign of the change and thus the occupancy changes over time.
In other words, if more people tend to get on at a given stop than get off, a longer stop here means a large positive change to the number of passengers.
Conversely, a stop which is associated mostly with disembarking, a long stop would mean a large negative change.

To test this, we waited at a single bus stop and counted the number of passengers entering and leaving each bus on a certain route, and the length of the stops (Fig.~\ref{fig:onestop}).
Data for this experiment was slower to collect, as the buses were spaced out.
This data suggests that this particular bus stop is predominantly a destination, rather than a starting point; longer stops here, then, would imply a net decrease in number of riders.
    
However, the data collected was insufficient to draw these conclusions; there can be significant variation depending on time and day of the week.
This data collection would need to be repeated at every bus stop, on a variety of days and times, in order to provide the required basis of prediction.
This clearly increases the time requirement to well beyond the scope of this study.
	
Thus, while the preliminary foray suggests that this approach could be successful, it is not feasible for this study.
We move on to our next approach.

\subsection{Promiscuous WiFi}
    
\subsubsection*{Packet Plot}
\minifig{packets}{Packets detected per 3-second bin: there are visible differences between spots, and frequencies are higher on busier stops.}{fig:buspackets}

In figure~\ref{fig:buspackets}, we plot the number of packets in three second bins.
Analyzing this figure, we see that there are visible differences between stops.
This roughly lines up to the expected traffic around each stop --- stops like the BCC, Hill, the ARC, and the entirety of College Avenue have a large amount of traffic, while stops near the Visitor's Center and Werblin have very little.

\subsubsection*{Unique Devices Plot}
\minifig{unique}{Unique MAC addresses detected per 3-second bin: by plotting devices instead of packets, we get cleaner frequencies.}{fig:busunique}

Network traffic is not a good indicator of occupancy, however, because it represents the amount of activity, rather than the number of actors; a single person transmitting a large amount of data may be interpreted as multiple people with the Packet Plot.
As such, our next step was to switch out ``number of packets'' for ``number of unique MAC addresses'' (Fig.~\ref{fig:busunique}).
The Router filter would also be used here, since we don't want to count routers among the number of riders.
The values on this plot differ noticeably between stops.
However, this is not due to occupancy; there are clearly not nearly that many people on the bus.
Instead, it seems that the shape of the plot is primarily influenced by the proximity to routers --- the portions of the bus loop by athletic fields and between campuses has nearly no data.

\subsubsection*{First-Last Grid Plot}
\bigfig{bus-grid}{First and Last times seen for each MAC address observed while on the bus route: clumps along the line $y=x$ are noise. Points further from the line represent devices present for longer durations.}{fig:busgrid}
\bigfig{bus-hmap}{First and Last times seen for each MAC address observed while on the bus route: similar to previous plot, but clumps are easier to identify.}{fig:bushmap}

In order to establish which devices were on the bus, we plotted the time of the first and last occurrence of each device (Fig.~\ref{fig:busgrid}).
Each point here represents a unique device, where the x-coordinate is the time of the first sighting, and the y-coordinate is the time of the last sighting.
Thus, the high concentration of data points along the line \(y=x\) represents the transient devices, most likely people the bus was passing by.
The clump of data points in the top left corner of the plot represents devices which were seen throughout the journey.
These are most likely our own devices, or stationary devices around the geographically identical first and last stops.

As a variation on this, we also plotted the same data as a heat map (Fig.~\ref{fig:bushmap}), such that higher concentrations of points are depicted by more vivid colors.
In the original grid plot, any number of datapoints at the same coordinates would be shown the same as a single datapoint.
This method allows us to see more easily how many points are in each area of the plot (between Rutgers Student Center and SAC, for example).

\subsubsection*{Segments Plot}
\minifig{segments}{Predicted number of passengers using the Segments approximation (blue) versus actual number of passengers (purple). Segments approximation calculated from the intersection of sets of observed devices at each stop.}{fig:segments}
The segments plot was created using the procedure for segments plot described in the Graphing section.
The plot (Fig.~\ref{fig:segments}) shows the actual number of bus passengers as we observed (the purple function) overlayed with the function that we calculated using the lengths of each intersect-set (the blue function).
One can see that the calculated function is accurate for the stops after Werblin.
The discrepancy between the actual and calculated number of bus passengers at Buell (label 1) may be influenced by the fact that we had started observation and recording at this location. In this method, more routers equals more stop-to-stop change.
This fact means the calculated function will appear more accurate in areas with more devices.
There are more routers around the College Avenue areas, so the calculated function appears more accurate at these stops in comparison to stops like Werblin, which have fewer surrounding devices and routers.
Due to the limitations of this method, we progressed to creating the vector plot.

\subsubsection*{Presence Vector Plot}
\minifig{vectors}{Predicted number of passengers using the Vector approximation (blue) versus actual number of passengers (purple). Verctor approximation calculated by summing number of devices between first and last sightings.}{fig:vector}

In an attempt to minimize noise from transient devices and to compensate for the lack of traffic in uninhabited areas, we constructed a plot (Fig.~\ref{fig:vector}) similar to the previous. 
For every address $i$ we encountered in the collection, we create a vector \(X^{(i)}= [ X_1 \cdots X_T ] \) such that
\begin{equation*}
  X_t^{(i)} = \begin{cases}
    1 & t_{\text{firstSeen}}^{(i)} \le t \le t_{\text{lastSeen}}^{(i)}\\
    0 & \text{otherwise}
  \end{cases}
\end{equation*}
with the sum of these vectors represented by $X$, such that \[X = \sum\limits_i X^{(i)}\]
The value at any point of $X$ indicates how many devices we can infer are presently on the bus.
Overlaid on this plot is the step function corresponding to the observed number of riders.
It is scaled differently, as there is approximately a three time difference between the devices seen and the people riding.

We see that there appears to be an excellent correlation here, with the exception of the 400-700 second segment. % uhh... what about 0-400? That's also pretty bad.
This segment corresponds to the entrance to the College Avenue campus, a highly concentrated area that appears to have contributed considerable noise.
We do see, however, that the two changes in occupancy in this region seem to correspond to peaks in the traffic --- this means that the traffic is probably higher than the actual by a constant value in that area, which we can account for in prediction. TODO: Roy's comment

\section{Future Work}

\bigfig{prob-hmap}{First and Last times seen for each MAC address during the class period}{fig:probhmap}

\bigfig{packethist}{Number of packets involving each MAC address present during the entirety of the class period}{fig:probhist}

\subsection{Continuation of Benchmark}
As mentioned earlier, the Stop Length Inference approach seemed promising, when looking at the simple quantity of traffic at a stop.
In order to utilize this, however, we would need to determine the tendency of each particular stop.
This means we would need to extensively analyze each stop individually over long periods of time.
Our preliminary research indicates this approach offers great opportunity for future research.
This research would, unfortunately, require far too much time and resources for four college students.

\subsection{Alternative Approach}
The largest problem with the interception of packets is the distance between the bus and access points.
Many, possibly most, of our detected packets were from passersby due to their relative proximity to routers; without a router to transfer a packet to, there cannot be a packet to detect.

One possible solution to this issue would be the installation of WiFi access points on Rutgers buses, using the standard Campus WiFi Access Point SSID.
Using the same SSID would cause student smartphones to automatically connect, without requiring any effort on the part of the students.
This would limit detection to Rutgers students, as well; no others would be able to connect to this AP.
This might, in fact, reduce accuracy on the College Avenue campus, but the overall impact would most likely be an increase in accuracy.
Optimally, these routers would be officially installed.
It is unlikely that Rutgers or First Transit would install these devices strictly for the purpose of detection, since our methods are still unproven.
Adding functionality in the way of internet, a router's expected purpose, could likely be done only through mobile data due to the bus routes going far outside the range of campuses.
Mobile data is very expensive, however, making this unlikely.
Any access points installed unofficially would be an explicit violation of the terms of use of the Rutgers networks, and, as such, even more infeasible.

Putting this burden of observation onto other students rather than officials would be much more manageable in terms of dealing with bureaucracy.
To do this, we could replace the laptop in our setup with a smaller, dedicated unit to be installed in the buses, to collect data for longer periods of time.
Running several of these units would allow us to collect enough data to make occupancy predictions.
However, the manpower required for this also makes this an improbable option.

\subsection{Potential Upcoming Problems}
Our current method of predicting bus occupancy is by monitoring unique MAC addresses.
This is already utilized in the real world to determine traffic patterns.
For example, the city of Houston uses the MAC addresses from Bluetooth devices to figure out the flow of traffic on city streets in their TranStar monitoring system. 
%http://appleinsider.com/articles/14/06/09/mac-address-randomization-joins-apples-heap-of-ios-8-privacy-improvements
Unfortunately, this technique may not be effective for much longer.
The next update for iOS, iOS 8, is expected to be released in fall 2014.
%http://www.apple.com/ios/ios8/?cid=wwa-us-kwg-features-com
This update will change how Apple devices broadcast their packets when searching for networks.
The device will randomly generate a MAC address during each round of scanning and will not display the real MAC address.
The reason for this update is to deter tracking by companies for the purpose of targeted advertisements and to provide higher security for consumers.
% http://appleinsider.com/articles/14/06/09/mac-address-randomization-joins-apples-heap-of-ios-8-privacy-improvements
Apple iPhones make up over 40\% of the smartphones owned in the US, and so this update would greatly impact the effectiveness of our program.
%http://appleinsider.com/articles/14/01/16/apples-iphone-now-represents-42-of-smartphones-owned-in-the-us---npd

\subsection{Alternative Application}
The methods we developed for the purpose of measuring bus occupancy can also be used for similar measurements in other settings.
To prove this, we measured occupancy in a classroom (fig.~\ref{fig:probhmap}).
Just as with the bus data, the points around \(y=x\) consist of transient devices.
The points in the clump to the top right is probably due to a buildup of students outside the lecture hall awaiting the following class.
The points in the clump to the top left represent the devices that were present for the duration of the capture.
Other scattered points may be noise, but could also be sleeping devices which only report themselves occasionally.

There were exactly 106 students in the lecture hall during this collection.
While judging exact cutoffs for the measurement area is of a complexity outside the scope of this project, a rough estimate of the clump in the top left gives about 120 addresses.
In order to judge how reliable this count is, we created a new type of plot, counting the packets including each of these long-present devices (fig~\ref{fig:probhist}).
This plot shows that most devices measured transmitted or received a reasonable quantity of packets -- only a few are particularly low on packet count.
This suggests that this approach does generate an accurate count of the lecture hall occupancy.

We can also observe that some devices are involved in a significantly larger quantity of packets.
This graph is generated with router filtering on, meaning it is unlikely that these are routers.
% More observations here?


%\section{Conclusion}
%	From the data above, we determined that the occupancy of a bus could be predicted by observing data traffic with some sensor placed on the bus.
%	We can filter out some of the irrelevant data by ignoring any signals below a certain strength.
%	This is the most feasible method, but requires external assistance with installation of devices.

\clearpage

%TODO: Add descriptions of difficulties encountered in doing each bullet point.
\appendix[Team Member Contributions]
Daniel Bordak
\begin{itemize}
\item Collected tcpdump and classroom data.
\item Created tcpdump output parser.
  Current implementation's design came from privacy concerns and speed requirements.
  Encountered difficulties in matching arbitrary MAC addresses and determining their type, as well as the aformentioned privacy concerns.
\item Optimized graph plotting functions for speed and maintainability.
  Usage of external libraries solved much of this.
  Integration of these libraries was particularly difficult, however, due to their demand of vectorized code.
\item Cowrote First-Last Grid Plot.
  Solved issues with axes.
\item Cowrote Segments Plot.
  Rewrote to better utilize Pandas features, dramatically increasing speed and accuracy.
\item Rewrote Unique Device Plot and Packet Plot to allow for different binsizes.
  Arbitrary binsizes led to later plot designs.
\item Wrote script to generate stop annotations, used on most plots.
  The desire to reduce redundancy led to the implementation of a common parser between both stop data and packet data.
\item Wrote sections of the report pertaining to the creation of plots.
  Difficulty encountered in remembering everything I did.% Heehee
\item Formatted report, including logfile syntax highlighting.
\end{itemize}
Erin Corrado
\begin{itemize}
\item Collected bus and classroom data
\item Cowrote First-Last Grid Plot.
  Difficulties in learning more advanced Python and learning how to utilize the Python libraries Numpy and Matplotlib.
\item Wrote length/occupancy correlation plot.
  Data was difficult to interpret at times.
  Data was time consuming to collect and limited.
\item Cowrote poster.
  Difficulty with trying to get the poster printed and transported to the proper place in time.
\item Cowrote Packet Histogram.
  Encountered difficulties with utilizing the new libraries used such as Pandas.
\item Proofread and edited report and graphs.
  Hardest part was keeping everyone's contributions in sync with one another.
  Also the group had to decide on style and formatting.
\end{itemize}
Revan Sopher
\begin{itemize}
\item Wrote Proposal
\item Collected tcpdump data
\item Wrote Raw Packet Plot
\item Wrote Unique Device Plot
\item Cowrote Segments Plot
\item Wrote Vector Plot
\item Wrote Report
\end{itemize}
Ashley Weaver
\begin{itemize}
\item Collected bus data
\item Cowrote Poster
\item Commented on report
\end{itemize}

\clearpage

\end{document}
