\section{Conclusion}

The Stop Length Inference approach supported correlations between the number of passengers on a bus and the length of time the bus stops.
Unfortunately, this method is impractical to utilize on a mass scale since collecting enough data to make conclusions about each stop would take an exorbitant amount of time and resources.
This approach may be possible with more people and time, but the time constraints make this method beyond the scope of this project.

The Promiscuous WiFi approach is much more efficient but has the problems of noise filtering and determining whether a device is actually present on the bus.
The methods that we used to analyze the WiFi data each had certain advantages and disadvantages.
The best way to use the data is to count the number of unique MAC addresses; the number of packets is not a reliable metric to determine the number of bus riders.
The First and Last time seen grid graphs reduce noise, but they do not represent the data in a fashion that is particularly easy to comprehend.
The Vector and Segment graphs are the most accurate in determining the number of bus occupants.
Both of these two graphs have specific sections where one is more accurate than the other.
Overall, the Vector and Segment approximations are the best way to analyze the data from the Promiscuous WiFi; therefore, these are the best approaches for predicting bus occupancy.
