\section{Future Work}
    
\subsection*{Continuation of Benchmark}
The Stop Length Inference approach seemed promising, but we would need to benchmark every stop several times in order to be able to know which direction of occupancy change a long stop indicates.
As such, there is substantial opportunity to complete this benchmarking in order to complete that approach.

\subsection*{Alternative Approach}
The largest problem with the interception of packets is outside noise.
To address this, we would like to install an access point on the bus, with the same SSID as the campus Wi-Fi so that smart phones would automatically connect.
We could then listen only to communications with this access point, eliminating outside traffic.
	
Optimally, the routers would be officially installed and serve mobile data.
This is naturally expensive and thus unlikely.
Any unofficial access points installed would be an explicit violation of the terms of use of the Rutgers networks, and thus unfeasible as well.

We could also replace the laptop in our setup with a smaller unit to be installed in the buses, to collect data for longer periods of time.
Running several of these units would allow us to collect enough data to make occupancy predictions.

\subsection*{Alternative Application}
While we initially set out to programatically predict the occupancy of buses, the Promiscuous Wi-Fi approach is also applicable to the determination of the occupancy of other scenarios.
To demonstrate, we ran tcpdump for twenty minutes during a lecture, and followed a similar analysis procedure.
The concept of "stops" doesn't apply to this situation; however, we can still derive interesting results.

\begin{figure*}[!t]
  \includegraphics[width=\textwidth]{lecture-grid}
  \label{fig:probgrid}
\end{figure*}

As before, the points around \(y=x\) consist of transient devices.
The points in the clump to the top right is probably due to a buildup of students outside the lecture hall awaiting the following class.
The points in the clump to the top left represent the devices that were present for the duration of the capture.
	
There were exactly 106 students in the lecture hall during this collection; while determining the exact boundaries is of a complexity outside the scope of this project, the clump in the top left contains approximately 100 points.
This suggests that this approach does generate an accurate count of the lecture hall occupancy.

\begin{figure*}[!t]
  \includegraphics[width=\textwidth]{packethist-sorted}
  \label{fig:probhist}
\end{figure*}

In this plot we see that we can partition devices into classes based on how frequent the packets.
While a small number of devices were generating enormous amounts of network traffic with millions of packets, others generated orders of magnitude fewer.
	
This suggests another interesting path of inquiry: monitoring the interest of an audience based on network traffic.
The assumption here is that a bored audience is more likely to cause network traffic by web browsing.
