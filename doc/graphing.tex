\section{Graph Production}

\subsection*{Preprocessing}

The raw output from running tcpdump contains many irrelevant details -- fields like the type of packet, or the antenna used.
As such, our first step in graph production was to preprocess this data, preserving only the information relevant to our goal.
The main data of interest was the list of MAC addresses seen in a given packet.
Each of these addresses can take one of four roles: Source, Destination, Transmitter, and Receiver.
Every packet will have some, but not all, of these types of addresses.
Since each line of tcpdump raw output represents a packet, our log files are first parsed line by line, each line individually stored as a Python Dict -- otherwise known as a Hash Map.
These Dicts hold the four potential addresses, the packet's strength, and what time it was sent at (in microseconds).
All the other information was disregarded; ignoring information during preprocessing is the simplest way of removing it from the data.

A few special considerations needed to be made during this step:
\begin{itemize}
\item Packets may not have a strength field.
  To make data handling easier later on, these are assigned a strength of zero.
  This is easier since it can be filtered out later simply by checking that the strength field is nonzero; a zero strength appearing naturally is impossible. 
\item Some lines contain only the address of a router, with no particular destination in mind.
  These are broadcast packets, sent to notify devices of the router's existence.
  Since we are monitoring personal devices, not routers, these lines are ignored, and the address identified as a router is flagged for future filtering.
\item Since MAC addresses are usually unique device identifiers, they could potentially be used to identify device owners.
  This kind of risk could subject our study to human subject regulations.
  In order to avoid this risk, addresses are saved into the Dicts as index numbers (starting from zero) rather than the actual address.
  This way, we can still differentiate between multiple devices without having to worry about our data being used for morally questionable purposes.
\end{itemize}
At the end of the parsing process, the Dicts are dumped into a JSON output file.
JSON is a open standard data interchange format.
Using JSON made utilizing the data much easier when it came to producing the graphs; most languages, Python included, have a standard library implementation for JSON, eliminating the effort spent in re-parsing the data. % TODO: Add example JSON output -- re-run raw2json with prettyprint on.

\subsection*{Filtering}
The preprocessing done above is only performed once per dataset.
Therefore, any sort of filtering that might change between two graphs must be done immediately before the construction of the graph.
We built four filters to be run at this stage:
\begin{itemize}
\item The Strength filter removes packets below a given strength threshold.
  For example, a strength filter of -50 would filter out packets of -51dB.
\item The Zero-Strength filter removes packets which have a strength field of zero.
  These packets correspond to tcpdump output lines missing a strength field, as mentioned above in the preprocessing section.
\item The Router filter removes any addresses which were flagged as routers during the parsing step.
  It also removes any packets which would no longer contain any addresses after the previous is performed.
\item The End Time filter removes any packets past a certain time.
  This is desirable when you only want to look at a part of a dataset.
  We never produced a corresponding Start Time filter since the need never arose.
\end{itemize}

\subsection*{Graphing}
For the construction of the graphs, a few additional libraries were needed: Numpy, Pandas, and Matplotlib.
Numpy is an advanced mathematical library for Python, and Pandas is a library built on top of Numpy to provide high-performance data structures.
We used these for data analysis.
Matplotlib is a 2D plotting library, which we used to generate the graphs. 
In order to simplify the graph production process, we made individual scripts for each graph type, plus a meta-script to make a united interface.
This meta-script reads a JSON file obtained from the preprocessing step, imports the data into a Pandas DataFrame (which can be thought of as a complicated 2D array), and sends this off to be filtered, as explained in the filtering step.
After the data has been filtered, it is handed off to one of the six plotting scripts:
\begin{itemize}
\item The Packet and Unique plots were created by generating an array of 3-second bins and adding the appropriate statistic to the bins.
  In the case of the packet plot, this was simply the number of packets found over those three seconds; the bin is simply an integer.
  In the case of the unique plot, however, the bin is a mathematical set holding the MAC addresses seen over the three second period.
  After the entirety of the three seconds are parsed, the bin is assigned the length of the set.
\item The First-Last Grid, Packet Histogram, and Vector plot were all created by checking the data set for each address.
  The first time each MAC address is encountered, it is added to a list with an initial value of the time it was seen as both its first and last time seen, and seen-count of one.
  Each additional time an address is seen, its last time seen value is updated to the current time, and its seen-count is incremented.
  In the Grid, the addresses are graphed using the first time seen as the x-value and the last time seen as the y-value.
  In the packet histogram, packets which fulfill the requirement of \(t_{first} + t_{max} - t_{last} < 120\) are selected, where \(t_{max}\) is the total observation time.
  These selected packets are then graphed with their y-values set as their seen-count.
  To make the graph easier to interpret, the y-axis was made to be logarithmic, and the addresses are sorted by their y-values.
  For the Vector plot, first an array of zeros was initialized with length equal to the length of observation in seconds.
  For each packet, every value between \(t_{first}\) and \(t_{last}\) was incremented.
  The resulting array was plotted as a step function, with the x-axis being its indices.
\item The segment plot is the most complicated plot we generated.
  First, a list of mathematical sets are generated, one for each stop.
  These sets are called the stop-sets.
  The stop-sets are populated with the MAC addresses seen during the corresponding stop.
  Next, a second list of mathematical sets are generated, again, one for each stop.
  These sets are called the intersect-sets.
  The intersect-sets are filled with the intersect between the stop-set of the same index value, and every other stop-set, combined by union.
  For instance, intersect-set 2 would contain
  \begin{multline*}
    \left(stop_2 \cap stop_0\right) \cup \left(stop_2 \cap stop_1\right) \cup \\
    \left(stop_2 \cap stop_3\right) \cup \dots \cup \left(stop_2 \cap stop_n\right)
  \end{multline*}
  The length of each intersect-set is considered to be the number of people on the bus at the stop.
  A special exception is made for the intersect of stop-set 0 and stop-set n.
  This set represents addresses present both at the beginning of the ride and the end.
  No normal student would do this -- this would consist only of ourselves and possibly devices present around this stop.
  As such, this set is instead subtracted from all other intersect sets; it is likely garbage.
  Afterward, we add a flat value of 2 to every stop, representing the two of us we filtered out.
\end{itemize}
Some graphs also had the actual number of people on the bus, as observed, overlaid.
This overlay was generated with a simple step function, using the value at n+1 as the value for the line between n and n+1.
